\section{Overview}

\subsection{Aim}
In this module we will implement a simple ImageJ macro to segment and analyze the blood vessel network of a subcutaneous tumor (see figure ~\ref{fig:bloodvessels}). The analysis is fully performed in 3D and possible strategies to extract statistics of the network geometry and interactively visualize the results are also discussed and implemented.

\subsection{Introduction}
\label{sec:mod8lab0}

Segmenting and extracting the geometry of the blood vessel network inside specific sub-regions of a tumor is a powerful investigation tool: The density of the vascularization and vessel branching points and the thickness of the vessels are for instance crucial age indicators to understand how the structure developed and possibly necrosed. With the help of a simple ImageJ macro these statistics can be extracted and the network 3D rendered with judicious color/transparency to provide insights on its organization.

\subsection{datasets}
The blood vessel datasets were acquired by a custom made (IRB Barcelona) macroSPIM allowing to image large (up to 1 cm), fixed and optically cleared samples (pieces of organs, tumors, whole organisms...). The preparation protocol and the imaging are similar to \cite{jahrling20093d}. For this project mice developing some specific tumors are injected a rhodamine-lectin construct to stain their blood vessels before sacrificing.
\textbf{Important note} : Two stacks cropped from the original dataset are provided, namely ''BloodVessels\_small.tif'' and ''BloodVessels\_med.tif''. It is \textbf{highly recommended} to first work on the smaller stack as processing time is not negligible. You may test the final ImageJ macro on the larger stack.