\section{Prefiltering to enhance filamentous voxels}

\subsection{Introduction}
The datasets used here exhibit a high contrast so that a simple intensity based thresholding is almost sufficient to distinguish tube from background voxels. However in case of higher noise and/or uneven sample staining one may need to filter the data prior to thresholding. A good criterion to follow for this operation is to notice that a voxel is part of a filament if there is one direction along which the intensity is quite constant (along the filament) and two perpendicular directions along which the intensities quickly drop (perpendicular to the filament). The ImageJ command \ijmenu{Plugins > Analyze > Tubeness} computes a metric reflecting to what extent a voxel and its local neighborhood fulfill this criterion. The implemented algorithm is based on \cite{Sato1998}.

\subsection{Workflow}
Select the output image of above section (''Closed.tif'').

\begin{description}

\item[Enhance filamentous voxels]\hfill\\
Use the \ijmenu{Plugins > Analyze > Tubeness} command on the data (after the morphological closing) and check the result for different "Sigma", which controls the size of a Gaussian filter that is applied before the actual ''Tubeness'' computation.
This Gaussian pre-filtering indirectly determines the size of the neighbourhood taken into account for computation of the local intensity distribution.

Sensible values are in the range of 6 to 8 micrometers but you can experiment with different values. It is in fact usually almost impossible to find a value that is optimal for both the smallest and the largest vessels. 

You will notice that the contrast is greatly enhanced after the filtering but voxels close to vessel branch points might be forced to zero as their neighborhood do not strictly follow the definition of being filamentous. If this problem is too pronounced it is possible to perform another pass of morphological closing after the pre-filtering to "repair" these gaps in the network.
\end{description}

\subsection{Generate an ImageJ macro script}
Implement a macro performing above operations.