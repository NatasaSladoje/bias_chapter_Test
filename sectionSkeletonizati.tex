\section{Skeletonization and analysis of the tubular network}

\subsection{Introduction}
In order to measure the network length and to count its branch points we will reduce the segmented tubes (which have a certain thickness) to their 1 voxel wide center lines. This process is called "Skeletonization". 
To identify the branch and end points of the skeletonized network one can use the following observations:

\begin{itemize}
\item End-point voxels have less than 2 neighbors.
\item Junction voxels have more than 2 neighbors.
\item Slab voxels (remaining voxels) have exactly 2 neighbors.
\end{itemize}
To perform these tasks we will use \ijmenu{Plugins > Skeleton > Skeletonize (2D/3D)} and \ijmenu{Plugins > Skeleton > Analyze Skeleton (2D/3D)} (see \cite{Arganda-Carreras2010}). Skeletonization is based on a specific connectivity. For 3D images ImageJ uses 26 neighbor per voxel by default.

\subsection{Workflow}

\begin{description}
\item[Skeletonization]\hfill\\
The label mask first needs to be binarized (all non zero voxels are objects) using \ijmenu{Image > Adjust > Threshold} with a lower threshold of 1 gray value. After this you can skeletonize the binary image using \ijmenu{Plugins > Skeleton > Skeletonize (2D/3D)}.  In order to visually check the skeletonization you may overlay the binary mask (or the original data) with the skeleton using for instance the command \ijmenu{Image > Color > Merge Channels...}. The original image might be assigned the gray channel and the skeleton the red channel.
\item[Skeleton analysis]\hfill\\
Use \ijmenu{Plugins > Skeleton > Analyze Skeleton (2D/3D)} to analyze the skeleton (for the moment leave all pruning options unchecked). Examine the image output, which has the following color-coding:
\begin{itemize}
\item End-point voxels: Gray value of 30, appearing blue.
\item Junction voxels: Gray value 70, appearing purple.
\item Slab voxels: Gray value 127, appearing red.
\end{itemize}
Examine the output table which not only contains the number of voxels falling into the three different classes but also the total length of the skeleton as well as their total number of end-points and junctions. If there are several disconnected skeletons in the image the statistics are reported for each of them. Observe that the number of junctions is smaller than the number of junction voxels, because at each junction there may be more than one voxel with more than two neighbors.
\item[Skeleton 3D visualization]\hfill\\
Visualize the analyzed skeleton in the 3D viewer. You may realize that it is not looking very nice, because it is only one voxel thick. Also the difference between slab, endpoint and branch voxels is not easy to see. 

\underline{Exercise}: Figure out a way to alter the skeleton for 3D visualization purposes.

\underline{Hint}: Change the value of the voxels by applying \ijmenu{Plugins > Process > Replace Value}; find an adequate combination of values and LUT, finally thicken the skeleton by dilating it in 3D. Be very careful when choosing the new values assigned to junction and end points as these voxels might be overwritten by close by slab voxels after dilation (local maximum operation).
\end{description}

\subsection{Generate an ImageJ macro script}
Implement a macro performing above operations.